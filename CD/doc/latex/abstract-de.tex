\section*{Zusammenfassung}

Die vorliegende Arbeit reiht sich in die Forschungsliteratur zu Online Communities, sozialer Netzwerkanalyse und medialer Inhaltsanalyse ein.
Sie nutzt einen frei verfügbaren Datensatz mit Kommunikationsdaten des Online Social Network (OSN) \emph{Reddit}, um das Interaktionsverhalten von Nutzern dieser Plattform zu untersuchen.
Auf Reddit haben Nutzer die Möglichkeit, Beiträge zu erstellen, zu bewerten und gegenseitig zu kommentieren.
Zudem können sie selbst eigene Communities gründen, sogenannte \emph{Subreddits}, was zu einer hohen Vielfalt an Themen führt, zu denen sie diskutieren und sich austauschen können.
Im Kern wird der Frage nachgegangen, ob das persönliche soziale Netzwerk eines Nutzers Einfluss darauf hat, wofür sich der Nutzer interessiert.  
Dazu wird mittels Latent Dirichlet Allocation ein statistisches Topic-Modell von Subreddits erstellt, das es ermöglicht, thematisch ähnliche Subreddits zusammenzufassen.
Zudem werden anhand von Nutzerkommentaren auf Reddit Interaktionsgraphen erstellt, die abbilden, wer mit wem kommuniziert.
Diese Interaktionsgraphen werden in einer Fallstudie mit Methoden der sozialen Netzwerkanalyse untersucht, um die zugrunde liegenden Strukturen besser verstehen zu können bzw. wie diese Netzwerkstrukturen mit dem Kommunikationsverhalten der Nutzer zusammenhängen.
Die Topic-Modelle stellen sich größtenteils als trennscharf heraus und bilden Themen ab, die man auf einem OSN wie Reddit erwartet.
Auch die Ergebnisse der sozialen Netzwerkanalyse stehen in Einklang mit dem, was vorherige Forschungsarbeiten festgestellt haben.
Zu klären bleibt die Frage nach der Güte des verwendeten Topic-Modells, welches das zentrale Element der Arbeit darstellt.
Ebenfalls zu leisten bleibt eine Untersuchung, welche die hier verwendeten Ideen und Methoden auf eine breite Nutzerbasis überträgt.