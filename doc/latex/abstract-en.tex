\selectlanguage{english}
\section*{Abstract}

We relate in parts to previous work on social network analysis and statistical language- and content-modelling.
To this end, we utilise a freely available dataset containing communication data collected from the online social network (OSN) \emph{Reddit} in order to analyse users' patterns of interaction.
Reddit facilitates creation of user content, and ranking and commenting thereof.
Furthermore, users are empowered to create their own communities or \emph{subreddits} in order to allow even further particularisation of interest.
We mainly ask what effect a user's immediate social network has on their interests and style of interaction.
Latent Dirichlet Allocation is used to build a topic model of subreddits, which enables us to summarise thematically similar subreddits.
Further, we collect comment threads on Reddit to connect users that communicate via comments into an interaction graph.
These interaction graphs can be made sense of with the tools and concepts of social network analysis to discover their structure.
A case study is conducted on three users to show the connection of these two approaches, the content- and the network-analytical.
The statistical topic models turn out to capture distinct topics that one would expect on an OSN such as Reddit.
Similarly, the results from conducting network analysis on users' interaction graphs confirms existing ideas in the literature.
However, validation of the topic model used in this work remains an open and important task.
Finally, conducting research using the ideas proposed here on a large-scale userbase is desirable in order to make sense of the complex structure of online, and essentially human social networks.

\selectlanguage{ngerman}