\pagestyle{empty} % Vorerst keine Seitenzahlen
\pagenumbering{alph} % Unsichtbare alphabetische Nummerierung

\begin{center}
\textsc{Ludwig-Maximilians-Universität München}\\
Department ``Institut für Informatik''\\
Professur für Computational Social Science and Big Data\\
Prof.\ Jürgen Pfeffer

\vspace{4.75cm}
{\large\textbf{Masterarbeit}}\vspace{.5cm}

{\Huge{}Not all those who wander are lost}\\\vspace{.5cm}
{\Large{}Dynamiken bei der Interessensentwicklung in Online Communities}\vspace{.75cm}

{\large Oliver Baumann}\\\href{mailto:baumanno@cip.ifi.lmu.de}{<baumanno@cip.ifi.lmu.de>}

\end{center}
\vfill

\begin{tabular}{ll}
Bearbeitungszeitraum: & 30.04.2018 bis 29.10.2018\\
Betreuer: & Dr.\ Mirco Schönfeld\\
Verantw. Hochschullehrer: & Prof.\ Jürgen Pfeffer
\end{tabular}

%______________________________________________________________________

\cleardoublepage
\section*{Zusammenfassung}

Die vorliegende Arbeit reiht sich in Forschungsliteratur zu interaktiven Tischen, interaktiven Arbeitsumgebungen,
gekrümmten Multitouch-Displays und indirekten Multitouch-Mappings ein. Anhand einer Nutzerstudie wird die Wirkung zweier indirekter
Eingabemodi auf den Nutzer untersucht. Dazu wurde für \emph{Curve}, ein interaktiver Tisch mit gebogenem Display,
eine prototypische Anwendung entwickelt, die entweder mit einer Maus oder über Multitouch-Gesten bedient werden kann.
Im Gegensatz zu isolierten Tasks ermöglicht die Anwendung den von einer Desktopumgebung gewohnten Arbeitsablauf. Das System
bietet für den Anwendungsfall "`Audio-Bearbeitung"' die Möglichkeit, in einem Audio-Sample zu navigieren und dieses
zu modifizieren. Die beiden Interface-Varianten wurden auf ihre Wirkung auf das Nutzererlebnis und ihre Eignung
zum Einsatz in interaktiven Arbeitsplätzen hin untersucht. Es wurde festgestellt, dass keine der beiden Varianten
dabei übermäßig gut oder schlecht abschneidet. Beide Eingabetechniken sind dabei ähnlich gut für den speziellen
Anwendungsfall geeignet. Ein Transfer zu anderen Einsatzmöglichkeiten schließt die Arbeit ab. Es sei darauf hingewiesen,
dass die in dieser Studie präsentierten Ergebnisse anhand einer kleinen Stichprobe ermittelt wurden und möglicherweise
nicht vollends generalisierbar sind.

\iffalse
\selectlanguage{english}
\section*{Abstract}

We relate in parts to previous work on interactive desks, interactive workspaces, bent multitouch-enabled displays
and indirect mappings for multitouch. Based on a user-study, we compare the effects of two interaction techniques on the user.
To this end, we implemented a prototypical application for \emph{Curve}, an interactive desk with a bent multitouch-display.
The user can interact with the application either via a mouse, or via multitouch-gestures.
In contrast to isolated tasks commonly used, our system enables a workflow comparable to that of a traditional desktop
environment. The system relates to the specific use-case of audio-editing and allows for navigating and manipulating an audio-sample.
Both interaction techniques presented here have been studied with regard to their effect on user experience and their
compatibility with being used in interactive workspaces. We conclude that neither of the two techniques out- or underperforms
and thus suggest that they are equally suitable for this particular use case. We end with suggesting other possible applications
of this setup, not restricted to any single use-case. We note, however, that due to the small sample size of our study, the
findings presented here might not be fully generalizable.
\selectlanguage{ngerman}
\fi
\cleardoublepage
\section*{Eidesstattliche Erklärung}

\selectlanguage{ngerman}

\noindent Ich erkläre hiermit, dass ich die vorliegende Arbeit
selbstständig angefertigt, alle Zitate als solche kenntlich gemacht
sowie alle benutzten Quellen und Hilfsmittel angegeben habe.

\vspace{7ex}
\noindent\makebox[9.3cm]{\dotfill}

\smallskip\noindent München, \today

%______________________________________________________________________

\cleardoublepage

\pagestyle{fancy}
\pagenumbering{roman} % Römische Seitenzahlen
\setcounter{page}{1}