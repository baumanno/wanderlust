\usepackage{booktabs}
\usepackage[ngerman,english]{babel} % deutsche Trennregeln, "Inhaltsverzeichnis" etc.
%\usepackage{mathptmx} % Times-Roman-Schrift (auch für mathematische Formeln)
% \usepackage{pdfpages}
% \usepackage{pgf}
% \usepackage{epstopdf}

\usepackage[labelfont=bf]{caption}
% \usepackage{changepage}
% \usepackage{subcaption}
% \usepackage{hanging}

\usepackage{array}

\usepackage{epigraph}
\renewcommand{\epigraphsize}{\small}
\setlength{\epigraphwidth}{.6\textwidth}

% custom imports
% multirows für Tabellen
% \usepackage{multirow}
% nice quotes
\usepackage[autostyle=true,german=quotes]{csquotes}

\usepackage[onehalfspacing]{setspace} % 1,5facher Zeilenabstand

% Zum Setzen von URLs
\usepackage{color}
\definecolor{darkred}{rgb}{0.25,0,0}
\definecolor{darkgreen}{rgb}{0,0.2,0}
\definecolor{darkmagenta}{rgb}{0.2,0,0.2}
\definecolor{darkcyan}{rgb}{0,0.15,0.15}

\usepackage{fancyhdr} % Positionierung der Seitenzahlen

\renewcommand{\headrulewidth}{0pt}

% Korrektes Format für Nummerierung von Abbildungen (figure) und
% Tabellen (table): <Kapitelnummer>.<Abbildungsnummer>
\makeatletter
\@addtoreset{figure}{section}
\renewcommand{\thefigure}{\thesection.\arabic{figure}}
\@addtoreset{table}{section}
\renewcommand{\thetable}{\thesection.\arabic{table}}
\makeatother

% include bibliography in ToC - special case for biblatex (bookdown doesn't handle this atm)
\let\oldpb\printbibliography
\renewcommand{\printbibliography}{\oldpb[heading=bibintoc,title=Referenzen]}

\let\oldpar\paragraph
\renewcommand{\paragraph}{\oldpar*}

\let\oldsubpar\subparagraph
\renewcommand{\subparagraph}{\oldsubpar*}

\renewcommand{\textfraction}{0.05}
\renewcommand{\topfraction}{0.8}
\renewcommand{\bottomfraction}{0.8}
\renewcommand{\floatpagefraction}{0.75}

\newcommand{\todo}[1]{{\textcolor{red}{\textbf{TODO:}~#1}}}

% römische numerale für Inhaltsverzeichnis; wird in index.Rmd zurückgesetzt
\fancyhead[LE,RO,LO,RE]{}
\fancyfoot[CE,CO,RE,LO]{}
\fancyfoot[LE,RO]{\roman{page}}
