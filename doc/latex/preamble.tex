\usepackage{booktabs}
\usepackage[ngerman,english]{babel} % deutsche Trennregeln, "Inhaltsverzeichnis" etc.
%\usepackage{mathptmx} % Times-Roman-Schrift (auch für mathematische Formeln)

\usepackage[labelfont=bf]{caption}
\usepackage[strict]{changepage}
\usepackage{placeins}

\usepackage{array}
\usepackage{calc}
\usepackage{epigraph}
\renewcommand{\epigraphsize}{\small}
\setlength{\epigraphwidth}{.6\textwidth}

% custom imports
% multirows für Tabellen
% \usepackage{multirow}
% nice quotes
\usepackage[autostyle=true,german=quotes]{csquotes}

\usepackage[onehalfspacing]{setspace} % 1,5facher Zeilenabstand

% Zum Setzen von URLs
\usepackage{color}
\definecolor{darkred}{rgb}{0.25,0,0}
\definecolor{darkgreen}{rgb}{0,0.2,0}
\definecolor{darkmagenta}{rgb}{0.2,0,0.2}
\definecolor{darkcyan}{rgb}{0,0.15,0.15}

\usepackage{fancyhdr} % Positionierung der Seitenzahlen

\renewcommand{\headrulewidth}{0pt}

% Korrektes Format für Nummerierung von Abbildungen (figure) und
% Tabellen (table): <Kapitelnummer>.<Abbildungsnummer>
\makeatletter
\@addtoreset{figure}{section}
\renewcommand{\thefigure}{\thesection.\arabic{figure}}
\@addtoreset{table}{section}
\renewcommand{\thetable}{\thesection.\arabic{table}}
\makeatother

% include bibliography in ToC - special case for biblatex (bookdown doesn't handle this atm)
\let\oldpb\printbibliography
\renewcommand{\printbibliography}{\oldpb[heading=bibintoc,title=Referenzen]}

\let\oldpar\paragraph
\renewcommand{\paragraph}{\oldpar*}

\let\oldsubpar\subparagraph
\renewcommand{\subparagraph}{\oldsubpar*}


\newlength{\bcorlength}
\setlength{\bcorlength}{12mm}              % binding correction
\newlength{\lmuprintspace}
\setlength{\lmuprintspace}{5mm}            % additional margin for title pages

\newcommand{\lmuemptypage}[1]{%
   \begin{adjustwidth}{-\oddsidemargin-1in+\lmuprintspace+\bcorlength}{-\evensidemargin-1in+\lmuprintspace}
      \vspace*{-\topmargin}\vspace{-1in}%
      \vspace{-\headheight}\vspace{-\headsep}%
      \vspace{-\topskip}%
      \vspace{+\lmuprintspace}
      \parbox[t][\textheight][t]{\linewidth}{%
         \parbox[t][\paperheight-3\baselineskip-\parskip][t]{\linewidth}{%
            \setlength{\parskip}{0pt}%
            \setlength{\parindent}{0pt}%
            \setlength{\parfillskip}{0pt plus 1fil}
            #1
          }%
      }%
      
   \end{adjustwidth}
}

\newcommand{\todo}[1]{{\textcolor{red}{\textbf{TODO:}~#1}}}
\usepackage{layout}
% römische numerale für Inhaltsverzeichnis; wird in index.Rmd zurückgesetzt
\fancyhead[LE,RO,LO,RE]{}
\fancyfoot[CE,CO,RE,LO]{}
\fancyfoot[LE,RO]{\roman{page}}
